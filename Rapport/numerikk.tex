% numerikk.tex
% Part of VertEgg documentation
%


\chapter{Numerical Methods}



The numerical solutions are computed on a fixed equidistant grid.  The
number of grid cells is denoted $N$ and the cell height is $\Delta
z$. The grid is staggered as shown in fig.~\ref{fig:grid}.  The
$z$-axis points upwards.
 
The cell interfaces, the \emph{flux points}\index{flux points}, are 
\begin{equation}\label{eq:flushpt}
  z_i = (1-i) \Dz \quad \text{for} \quad i = 1, \dots, N+1 .
\end{equation}
The cell centers, the \emph{concentration points}\index{concentration
points} or the \emphi{egg points}, are
\begin{equation}\label{eq:eggpt}
  \bar{z}_i = \frac{z_i + z_{i+1}}{2} = (\half-i) \Dz
            \quad \text{for} \quad i = 1, \dots, N .
\end{equation}

\begin{figure}[h]
\begin{center}
\setlength{\unitlength}{1mm}
\begin{picture}(65,20)
  \multiput(10.0, 8.5)(20,0){3}{\line(0, 1){3}}
  \put(20.0,10.0){\makebox(0,0){$\times$}}
  \put(40.0,10.0){\makebox(0,0){$\times$}}
  \put(10.0, 6.0){\makebox(0,0){$z_{i+1}$}}
  \put(30.0, 6.0){\makebox(0,0){$z_{i}$}}
  \put(50.0, 6.0){\makebox(0,0){$z_{i-1}$}}
  \put(20.0, 6.0){\makebox(0,0){$\bar{z}_{i}$}}
  \put(40.0, 6.0){\makebox(0,0){$\bar{z}_{i-1}$}}
  \put(10.0,14.0){\makebox(0,0){$F_{i+1}$}}
  \put(30.0,14.0){\makebox(0,0){$F_{i}$}}
  \put(50.0,14.0){\makebox(0,0){$F_{i-1}$}}
  \put(20.0,14.0){\makebox(0,0){$\phi_{i}$}}
  \put(40.0,14.0){\makebox(0,0){$\phi_{i-1}$}}
  \put(0,10){\vector(1,0){60}}
\end{picture}
\end{center}
\caption{Horizontal view of the vertical grid}\label{fig:grid}
\end{figure}

The variables are discretized in the following ways. The egg
concentration $\phi_i$ represents the mean concentration in
cell $i$, that is
\begin{equation}\label{eq:cellav}
  \phi_i \Dz = \int_{z_{i+1}}^{z_i} \!\phi(z)\,dz, 
                 \quad i = 1, \dots ,N .
\end{equation}
The $\phi_i$ values may be considered living on the egg points
$\bar{z}_i$ as in fig.~\ref{fig:grid}. As the flux values are computed
from the vertical velocity and the eddy diffusivity,
$F_i$, $w_i$ and $K_i$ are taken as point values on the flux points
$z_i$ for $i = 1, \dots, N+1$. The source terms $P_i$ and $\alpha_i$
are cell averages and live on the egg points.


\section{The Transient Problem}\label{sec:numtrans}

The numerical solution of the convection-diffusion
equation~(\ref{eq:difflaw}) are covered by several authors.  A classic
source is Roache \shortcite{roac72}. A newer source is chapter 9 in
the book by Fletcher \shortcite{flet91}. Recently a whole book, edited
by Vreugdenhil and Koren \shortcite{vreu93}, has been devoted to this
equation.  For conservative methods, as will be used here, the book by
LeVeque \shortcite{leve92} is also recommended.

For our problem, convection and diffusion of concentration of fish
eggs, some properties of the numerical method is important.  Fish eggs
are usually found in a subrange of the water column with values close
to zero outside this range. And of course a real concentration do not
have negative values.  The method must therefore be
\emph{positive}, that is it must not create
any negative values.

In the absence of source and sink terms, eggs are not created or
destroyed. The vertical integrated concentration is therefore constant
in time as shown in section~\ref{sec:vertint}.  The numerical method
should have the same property, it should be \emph{mass
conserving}\index{mass conserving method}. 

Of course the numerical solution should be as close as possible to the
real solution. That is the \emph{accuracy} of the method should be
good. This concept includes low artificial diffusion and no or very
limited wiggles development.

A convection -- diffusion problem is characterised by the following
non-dimensional numbers, all defined at the interior flux points,
$z_i$ for $i = 2, \dots, N$.
\begin{align}
  & \text{Courant number}         & c_i & = w_i\frac{\Dt}{\Dz} \\
  & \text{Diffusive parameter}    & s_i & = K_i \frac{\Dt}{\Dz^2} \\
  & \text{Cell Peclet number}     & \Pcell_i & = 
       \frac{\abs{w_i} \Dz}{K} = \frac{\abs{c_i}}{s_i}
\end{align}
\index{Courant number}\index{diffusive parameter}\index{cell Peclet number}
If the coefficients $w$ and $K$
are constant, the subscripts are simply dropped.

For fish eggs, the terminal velocity is in the order of a couple of
millimetres per second. The velocity can be positive (\emph{pelagic}
eggs)\index{pelagic eggs}, negative (\emph{benthic})\index{benthic
eggs} or change sign some place in the water column
(\emph{mesopelagic})\index{mesopelagic eggs}. The vertical eddy
diffusion show a very large range of variation, from more than
$10^{-2}$ \sqmps\ in the upper mixing layer to $10^{-5} \sqmps$
in the pycnocline layer.  With a grid size of the
order of one meter, the Cell Peclet number $\Pcell$ take values from
$10^{-1}$ to $10^2$. This means that the process may be dominated by
diffusion in the upper layer and convection below.

The numerical methods considered here are \emph{finite difference} or
really \emph{finite volume} schemes given in \emph{conservation
form}\index{conservation form} (or \emph{flux
formulation}\index{flux formulation}).  These methods are
conceptually natural, working directly with the integral form of the
conservation law~(\ref{eq:intlaw}).  More precisely, the
equation~(\ref{eq:intlaw}) is approximated by
\begin{equation}\label{eq:numlaw}
  (\phi_i^{n+1} - \phi_i^n) \Dz = 
           (F_{i+1}^n - F_i^n) \Dt + Q_i^n \Dz \Dt ,
           \quad i = 2, \dots, N
\end{equation}
where the superscripts indicate the time step. 
The boundary conditions (\ref{eq:boundcond}) simply become
$F_1 = F_{N+1} = 0$. The various methods
differ in their estimation of the total flux function $F_i = \Fconv_i
+ \Fdiff_i$ and the source term $Q_i$.  One advantage of this flux
formulation is that mass conservation is automatically fulfilled, as
the fluxes cancel out during vertical integration.

Without source term the total concentration is conserved. Therefore
the local concentration values can not become arbitrary large, unless
there are negative values in some other cells. In other words,
positivity is a sufficient (but not necessary) condition for stability
of such schemes.

In the following several methods are described.  They are implemented in
the toolbox as \edbi{ftcs}, \edbi{lwendrof}, \edbi{upstream},
\edbi{posmet}, and \edbi{minlim} respectively. Their performance
will be studied in the example section~\ref{sec:sens}. Several methods
for the same problem causes a new problem, which method to choose. This
depends on the range of $\Pcell$ values in the problem. A general
advice is to try Lax-Wendroff first. This is an accurate and fast method
when applicable. If oscillations occur, go for the flux-limited
methods or use higher spatial resolution.  In both cases more computing
time is required. 



\subsection{Linear conservative schemes}\label{seq:linsch}

The linear conservative schemes considered here estimates the total 
flux function in the form 
\begin{equation}\label{eq:alphabeta}
  F_i = \alpha_i \phi_i + \beta_i \phi_{i-1} ,
\end{equation}
where the $\alpha_i$-s and $\beta_i$-s are independent of 
the $\phi_i$-s. Sometimes this will be used in non-dimensional form
\begin{equation}\label{eq:ab}
  F_i = \frac{\Dz}{\Dt} (a_i \phi_i + b_i \phi_{i-1}) .
\end{equation}
Some general theory for this kind of numerical schemes is presented
in Appendix~\ref{app:lincons}


\subsubsection{The FTCS scheme}\index{FTCS scheme}

The simplest scheme is often called FTCS (forward time central space)
after the differencing. This scheme is described for instance in
chapter 9.4 in \cite{flet91}. Here both flux
components are centred around $z = z_i$,
\begin{equation}
  \Fconv_i = w_i \frac{\phi_{i-1} + \phi_i}{2}
\end{equation}
\begin{equation}\label{eq:centdifflux}
  \Fdiff_i = - K_i \frac{\phi_{i-1} - \phi_i}{\Dz} .
\end{equation}
In the notation of equation (\ref{eq:alphabeta}) the method is given by
\begin{equation}
   \alpha_i = \frac{1}{2} w_i + \frac{K_i}{\Dz} , \quad 
   \beta_i  = \frac{1}{2} w_i - \frac{K_i}{\Dz} ,
\end{equation}
and non-dimensionalised by
\begin{equation}
   a_i = \frac{1}{2} c_i + s_i , \quad 
   b_i = \frac{1}{2} c_i - s_i .
\end{equation}

The following positivity condition for the FTCS scheme follows from
the general condition in Appendix~\ref{app:pos}.
\begin{gather}
     \abs{c_i} \le 2 s_i,  \quad i = 2, \dots, N \\
      s_2 \le 1 + \half c_2 \\
  s_i + s_{i+1} \le 1 + \half (c_i - c_{i+1}), \quad i = 2, \dots, N-1 \\ 
      s_N \le 1 - \half c_N
\end{gather}
The first condition can be restated as $\Pcell \le 2$.

With constant coefficients, the positivity conditions reduce to
\begin{equation}
  \abs{c} \le 2 s \le 1,
\end{equation}
which is stronger than the stability condition
\begin{equation}
  c^2 \le 2 s \le 1 .
\end{equation}

The numerical ``diffusion'' is negative, 
\begin{equation}
  K_{num} = - \half w^2 \Dt
\end{equation}
which is negligible if $c^2 \ll 2s$.

This method is implemented in the toolbox as the function \edbi{ftcs}.

\subsubsection{The Lax-Wendroff Scheme}\index{Lax-Wendroff scheme}

An alternative is the Lax-Wendroff scheme as analysed for instance in
\cite{vreu93b}. The scheme compensates for the negative numerical
``diffusion'' in FTCS.  The diffusive flux is still given by
equation~(\ref{eq:centdifflux}) but the convective flux is modified
\begin{equation}\label{eq:lwflux}
  \Fconv_i = w_i \frac{\phi_{i-1} + \phi_i}{2}
      - \half w_i^2 \frac{\Dt}{\Dz} 
       (\phi_{i-1} -\phi_i) .
\end{equation}
In the notation of equation (\ref{eq:alphabeta}) the Lax-Wendroff
method is given by
\begin{equation}
   \alpha_i = \frac{1}{2} w_i(1+w_i\frac{\Dt}{\Dz}) + \frac{K_i}{\Dz} , \quad 
   \beta_i  = \frac{1}{2} w_i(1-w_i\frac{\Dt}{\Dz}) - \frac{K_i}{\Dz} ,
\end{equation}
and non-dimensionalised by
\begin{equation}
   a_i = \frac{1}{2} c_i(1+c_i) + s_i , \quad 
   b_i = \frac{1}{2} c_i(1-c_i) - s_i .
\end{equation}

From Appendix~\ref{app:pos} the conditions for positivity are
\begin{gather}
     \abs{c_i} \le 2 s_i + c_i^2,  \quad i = 2, \dots, N \\
     s_i + s_{i+1} \le 1 + \half (c_i - c_{i+1}) - \half(c_i^2 + c_{i+1}^2),
               \quad i = 2, \dots, N-1.
\end{gather}
The first condition can be restated as 
\begin{equation}\label{eq:lwpcell}
    \Pcell \le \frac{2}{1-\abs{c_i}} .
\end{equation}

With constant coefficients, the positivity
conditions reduce to
\begin{equation}
 \abs{c} \le c^2 + 2 s \le 1,
\end{equation}
which is stronger than the stability condition
\begin{equation}
  c^2 + 2 s \le 1 .
\end{equation}

There is no second order numerical diffusion in the transient
solution.  Wiggles will not occur if the positivity
condition~(\ref{eq:lwpcell}) is fulfilled.
This method is implemented in the toolbox as the function \edbi{lwendrof}.

\subsubsection{The Upstream Scheme}\index{upstream scheme}

A non-centred alternative is the upstream method.  This is the method
used by West\-g{\aa}rd \shortcite{west89}.  The upstream scheme is
studied in all books on the subject.  The same diffusive flux
(\ref{eq:centdifflux}) is used but the convective flux is estimated on
the inflow side.
\begin{equation}\label{eq:usflux}
  \Fconv_i = w_i^+ \phi_i + w_i^- \phi_{i-1}
\end{equation}
where 
\begin{equation}
   x^+ = \half (x + |x|) = 
       \begin{cases}
          x& \text{if $x \ge 0$},\\
          0& \text{if $x  < 0$}.
       \end{cases}
\end{equation}
and $x^- = x - x^+$.
In the notation of equation (\ref{eq:alphabeta}) the method is given by
\begin{equation}
   \alpha_i =  w_i^+ + \frac{K_i}{\Dz} , \quad 
   \beta_i  =  w_i^- - \frac{K_i}{\Dz} ,
\end{equation}
and non-dimensionalised by
\begin{equation}
   a_i = c_i^+ + s_i , \quad 
   b_i = c_i^- - s_i .
\end{equation}

From Appendix~\ref{app:pos} the positivity condition is simply
\begin{equation}\label{eq:uspos}
  c_i^+ - c_{i+1}^- + s_i + s_{i+1} \le 1 .
\end{equation}
With constant coefficients this reduces to
\begin{equation}
  \abs{c} + 2 s \le 1 .
\end{equation}
which is also the stability condition.

The numerical diffusion in the upstream scheme may be quite large
\begin{equation}
  K_{num} = \half w \Dz (1-\abs{c})
\end{equation}
This is negligible if $K_{num} \ll K$, or equivalently
\begin{equation}
  \Pcell \ll \frac{2}{1 - \abs{c}} .
\end{equation}

The upstream scheme is implemented in the toolbox as the function
\edbi{upstream}.

\subsection{Nonlinear methods}\label{sec:nonlin}\index{non
linear method}

The Lax-Wendroff scheme is second order in space but may develop
wiggles and negative concentration values. The upstream method is
positive but may be too diffusive. These are fundamental problems with
linear schemes.  Several nonlinear schemes has been developed to
overcome these problems.  A general idea is to combine the
Lax-Wendroff and upstream methods to produce positive schemes with low
numerical diffusion. Overviews of such methods are given by LeVeque
\shortcite{leve92} and in the collection \citep{vreu93}.

Egg distribution problems are somewhat non-symmetrical. It is
important to have high accuracy near maxima where most of the eggs are
found. Local minima are less interesting and they occur more seldom in
the interior of the water column. For instance, with the steady state
solution (\ref{eq:sstatesol}) without source terms, a local minimum in
the interior can occur only in static instable situations when the egg
is lighter than the water above and heavier saline than the water
below.

The methods below are not total variation diminishing (TVD) as some of
the more advanced non-linear methods.  Instead of a finely tuned
linear combination, the schemes use simple on/off mechanisms to switch
between the Lax-Wendroff and upstream fluxes.  On the other hand, the
asymmetry above is exploited. The schemes have high accuracy at maxima
because the unmodified Lax-Wendroff flux is used in this situation.



\subsubsection{The positive method}\index{positive scheme}

This is a simple scheme, mostly using the Lax-Wendroff
fluxes but limiting to the upstream fluxes where Lax-Wendroff produces
negative concentration values. The algorithm consists of three steps,
\begin{enumerate}
\item Compute the Lax-Wendroff fluxes including diffusivity by
   formul{\ae} (\ref{eq:lwflux}) and (\ref{eq:centdifflux}).
\item Apply the fluxes to compute a test distribution
\begin{displaymath}
  \phi_i^{test} = \phi_i - \frac{\Dt}{\Dz}(F_i - F_{i+1}) .
\end{displaymath}
\item Where $\phi_i^{test} < 0$, recompute $F_i$ and $F_{i-1}$ by the
  upstream formulation (\ref{eq:usflux}) and diffusion (\ref{eq:centdifflux}).
\end{enumerate}

Positivity of the upstream scheme or Lax-Wendroff scheme implies
positivity of the flux limited scheme. The positivity
condition~(\ref{eq:uspos}) is therefore a sufficient (but not
necessary) condition for positivity of the combined scheme.

The scheme is implemented in the toolbox as the function \edbi{posmet}.



\subsubsection{The minimum limiting method}
\index{minimum limiting scheme}

Although the method above is positive, it may create wiggles around a
positive value.  To improve on this situation a slight variant is
proposed.  The new criteria for switching from Lax-Wendroff to
upstream flux is the occurrence of a local minimum.  As the first
negative value must be a local minimum prevents the method from
creating negative values (if the upstream scheme is stable).

Steps 1 and 2 in the algorithm are identical to the positive scheme.
The third step is modified to 
\begin{quote}
  3'. Where $\phi_i^{test} < \min (\phi_{i-1}^{test},\phi_{i+1}^{test})$, 
      recompute $F_i$ and $F_{i-1}$ by the
      upstream formulation (\ref{eq:usflux}) and diffusion 
      (\ref{eq:centdifflux}).
\end{quote}

This scheme is implemented in the toolbox as the function \edbi{minlim}.

\subsection{The Source Term}

Neglecting the fluxes for the moment, the transport equation
\pref{eq:difflawfull} reduces to an ordinary differential equation
\begin{equation}\label{eq:noflux}
  \Od{\phi}{t} = P - \alpha \phi .
\end{equation}
The numerical scheme \pref{eq:numlaw} reduces to
\begin{equation}\label{eq:numnoflux}
  \phi_i^{n+1} - \phi_i^n = Q_i^n \Dt .
\end{equation}
The exact solution of (\ref{eq:noflux}) with
constant coefficients is
\begin{equation}
  \phi^{n+1} = \e^{-\alpha \Dt} \phi^n + 
                 (1 - \e^{- \alpha \Dt}) \frac{P}{\alpha} .
\end{equation}
This is in the form \pref{eq:numnoflux} with
\begin{equation}\label{eq:kildediskret}
  Q_i = (P_i - \alpha_i \phi_i) 
         \frac{1-\e^{- \alpha_i \Dt}}{\alpha_i \Dt} .
\end{equation}
This scheme will never produce negative concentration values.

Combining this with the convection-diffusion solvers above is
straightforward. The combined positivity condition becomes more
restrictive as discussed in Appendix~\ref{app:posloss}. 
In general the number $1$ appearing as upper limit must be replaced
by $\exp(-\alpha \Dt)$.  For instance, with constant coefficients
the positivity conditions for the familiar linear schemes become
\begin{align}
  & \text{FTCS}         & \abs{c} \le 2 s & \le \e^{-\alpha \Dt} , \\
  & \text{Lax-Wendroff} & c^2 + 2 s       & \le \e^{-\alpha \Dt} , \\
  & \text{Upstream}     &  \abs{c} + 2 s  & \le \e^{-\alpha \Dt} .
\end{align}


  
\section{The Stationary Problem}\label{sec:statprob}

In many problems only the steady state solution is needed.  Also for
transient problems, a fast and accurate way of reaching the stationary
solution is of interest.

If the coefficients $w$ and $K$ are constant, the exact
solution~(\ref{eq:eggsact}) can be used directly. With negative
velocity and low diffusivity, $m = w/K \ll 0$, the formula may overflow.  This
can be avoided by using the symmetry relation~(\ref{eq:symmetry}) for
negative $m$.  This solution is computed by the function \edbi{eggsact}
in the toolbox. For the discretized solution to have the correct
vertical integral and be comparable to the numerical solutions, cell
averages
\begin{equation}
  \phi_i = \frac{\Phi}{\Dz}
           \frac{\e^{m z_i} - \e^{m z_{i+1}}}{1 - \e^{-mH}} 
\end{equation}
are better. This is also computed by \edbi{eggsact}.


General $w$ and $K$ can be regarded as constant on the grid cells.
Suppose $w$ and $K$ and their quotient $m$ live on the flux points as
in the transient problem. Form the cell average $\bar{m}_i = (m_i +
m_{i+1}) / 2$. The cell averaged solution becomes
\begin{equation}
  \bar{\phi}_i = 
     \frac{C_i}{\Dz} 
     \frac{\e^{\bar{m}_i z_i} - \e^{\bar{m}_i z_{i+1}}}{\bar{m}_i}
\end{equation}
where the $C_i$-s are computed by continuity
\begin{equation}
  \phi_i(z_i) = \phi_i(z_{i+1})
\end{equation}
and vertical integral
\begin{equation}
  \Dz \sum_{i = 1}^N \bar{\phi_i} = \Phi
\end{equation}
With low mixing and high sinking velocity the exponential terms may
overflow. The computation of the $C_i$-terms may overflow.  The last
problem is overcome by working with the logarithms. For the first
problem the $C_i$-terms are renormalised by finding a suitable value for
$C_1$. The method is implemented in the toolbox as the function
\edbi{sstate}.


%\section{Particle Methods}

