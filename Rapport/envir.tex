% envir.tex
% part of the VertEgg documentation
%
\chapter{Working in the MATLAB/Octave Environment}

MATLAB\index{MATLAB} and Octave\index{Octave} offer command-line
oriented interactive environments for numerical computations, data
analysis and visualisation. They also provide high level programming
languages. These programmes, called \emph{M-files}\index{M-files}, come
in two flavours, \emph{scripts} which are sequences of commands
executed in batch, and \emph{functions} which can take and return
arguments. Functions can also have local variables.  M-files may be
viewed as extensions of the basic system.  A collection of M-files for
a specific area is called a \emph{toolbox}\index{toolbox}.  For marine
research for example, there exists a free toolbox for analysis of
physical oceanographic data, named SEAWATER\index{SEAWATER}
\citep{morg94}.


The basic environments and the programming languages are quite
compatible. They provide the same basic data structure, arrays of
dimension up to two (scalars, vectors, matrices).  This makes it
possible to write M-files that work equally well for both systems. The
systems have simple but powerful systems for online help. 
The main difference between the systems is the much stronger graphic
capabilities of MATLAB.



Octave is documented in \citep{eato95}.  MATLAB is documented in user
and reference manuals, there is also a nice little primer \citep{sigm94}.

\section{Implementation of the VertEgg Toolbox}

The geometric setup of the problem is basic for all other
work. This setup is therefore given by a set of global variables
which can be accessed in the work space and by all functions in the toolbox.

They are declared by the script \edbi{ve\_init} which
contain the following statements.
\begin{verbatim}
  global Ncell   % Number of grid cells 
  global dz      % Vertical step size  [m] 
  global Hcol    % Depth of water column [m]
  global ZE      % Ncell-vector of egg-point depths [m]
  global ZF      % Ncell+1-vector of flux-point depths [m] 
\end{verbatim}
\index{Ncell@\texttt{Ncell}}\index{dz@\texttt{dz}}
\index{Hcol@\texttt{Hcol}}\index{ZE@\texttt{ZE}}
\index{ZF@\texttt{ZF}}



Given the depth $\edb{Hcol} = H$ and the grid height $\edb{dz} =
\Delta z$, the rest of the values can be calculated.
This is done by the function \edb{ve\_grid(Hcol,dz)}.
The vector $\mbox{\edb{ZE(i)}} = \bar z_i$ and 
$\mbox{\edb{ZF(i)}} = z_i$ as defined
in chapter ....

The commands \edb{ve\_init} and \edbi{ve\_grid} can be reissued at any
time. The only limitation is that \edb{ve\_init} must be called once before
\edb{ve\_grid} and nearly all other VertEgg statements.

Many (but not all) of the commands have names starting with \edb{ve\_},
to not be mixed up with commands in the basic systems or other
toolboxes. This may be done more consequently in future versions.

A discretized egg distribution is represented by a column vector
of length \edb{Ncell}. The elements are cell averages.
A matrix of size $\edb{Ncell} \times \edb{n}$
can be viewed as a collection of \edb{n} distributions. The commands
in VertEgg and most general MATLAB/Octave commands work on
distributions as a whole. For example, plotting the vertical profile
is done by the command \edb{plot(A,ZE)}.

The vector languages give a nice environment for working on 1D
problems such as vertical distributions. For larger problems in 2D or
3D the limitation to 2D array\footnote{The restriction to 2D arrays is
removed in MATLAB version 5} and the performance penalty of an
interpreted language make MATLAB/Octave less interesting.
Even for 1D problems there are certain tricks that must be used to
gain acceptable performance. The most important is to use vector
constructs instead of loops. For instance, to compute the central
diffusive flux approximation (\ref{eq:centdifflux}) in Fortran 77
requires the loop.
\begin{verbatim}
      DO 11 I = 2, NCELL
        FDIFF(I) = - K(I) * (A(I-1) - A(I))
   11 CONTINUE
\end{verbatim}
A similar loop is also possible in MATLAB/Octave,
 but much better performance is achieved by the vector subscript
\begin{verbatim}
  Fdiff(2:Ncell) = - K(2:Ncell) * (A(1:Ncell-1) - A(2:Ncell)).
\end{verbatim}
Under Octave, the variable \edb{do\_fortran\_indexing} should be true.
This gives better compatibility with MATLAB for vector subscripts.





