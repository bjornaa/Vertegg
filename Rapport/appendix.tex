
\chapter{Installation}

\section{Availability of the software}

In the spirit of free exchange of scientific ideas, the VertEgg toolbox
is free software. The software is available by anonymous ftp from
the cite \edb{ftp.imr.no} in the directory \edb{/pub/software/VertEgg}.

The software  can be freely redistributed and modified, the only
restriction is that it can not be included in commercial or 
shareware software. The author and the Institute of Marine Research
can of course not give any guarantie that VertEgg behaves as it should.

If you download and in particular if you use VertEgg the author would
like to be informed. Use \edb{e-mail:bjorn@imr.no}. The VertEgg
software and/or this report should be refered to when publishing results
obtained by the use of VertEgg.

\section{Matlab under Microsoft Windows}

Matlab\index{Matlab} is a commercial product of The MathWorks, Inc., The
official web-page is \verb+http://+ \verb+www.mathworks.com/+. Matlab is
available for PC, Macintosh and UNIX workstations.  Vertegg has been
developed and tested on version 4.2c.1 on PCs running Windows 3.11 and
Windows 95. 

The VertEgg toolbox consist of two directories, the toolbox and the
example directory. The toolbox directory can be placed anywhere, but a
natural place is \verb+C:\MATLAB\+ \verb+TOOLBOX\LOCAL\VERTEGG+.
Matlab must be told about the location of the toolbox. This is best
done by adding the the directory to the \edb{matlabpath} in the master
startup file \verb+C:\MATLAB\MATLABRC.M+ or your own startup file.
Alternatively, the directory can be added by the \edb{path} command in
an interactive session.  The examples directory is not required, and
can be placed anywhere.


\section{Octave under UNIX}

A short description of Octave\index{Octave} is taken from the README-file,
\begin{quotation}
   Octave is a high-level language, primarily intended for
   numerical computations. It provides a convenient command line
   interface for solving linear and nonlinear problems
   numerically. 

   Octave is free software; you can redistribute it and/or modify
   it under the terms of the GNU General Public Licence as
   published by the Free Software Foundation.
\end{quotation}

Octave is developed by J.W.~Eaton from the University of Wisconsin.
The official web-page is \edb{http://bevo.che.wisc.edu/octave.html}.
The program is available by anonymous \edb{ftp} from the official
site, \edb{bevo.che.wisc.edu} in the directory \edb{/pub/octave}.
Source code and binaries for many UNIX workstations are available. The
current version (August 1995) is 1.1.1.


Octave uses gnuplot\index{gnuplot} as its graphic library. The VertEgg toolbox
will work without the graphics, but most of the examples will not.
To do any serious work with Octave, gnuplot must be installed before
Octave. Gnuplot is also a useful program by itself.

From the frequently asked questions (FAQ) about gnuplot,
\begin{quotation}
   Gnuplot is a command-driven interactive function plotting
   program. It can be used to plot functions and data points in
   both two- and three- dimensional plots in many different
   formats, and will accommodate many of the needs of today's
   scientists for graphic data representation. Gnuplot is
   copyrighted, but freely distributable; you don't have to pay
   for it.
\end{quotation}

Gnuplot has many authors and is available on all platforms.
The official www homepage for gnuplot is 
\verb+http:+\verb+//www.cs.+\verb+dartmouth.edu+\verb+/gnuplot_info.html+, the
program is available by anonymous \edb{ftp} from
\edb{ftp.dartmouth.edu} in the directory \edb{/pub/gnuplot} and many
other cites. The current version (August 1995) is 3.5 but
beta-releases of version 3.6 are updated frequently.

The natural place for the VertEgg toolbox directory is
\verb+/usr/local/lib/octave/site/+ \verb+m/vertegg+ where it
automatically will be found by Octave. The examples directory can be
placed anywhere.


\chapter{Mathematical digressions}

\section{Exact solution of transient problem with constant
  coefficients}



If the eddy diffusivity $K$ and the egg velocity $w$ are constant and
the source term $Q$ is zero, the convection-diffusion equation
(\ref{eq:difflaw}) can be solved analytically.

The problem is,
\begin{equation}\label{eq:constpde}
  \phi_t + w \phi_z - K \phi_{zz} = 0
\end{equation}
with boundary conditions
\begin{equation}
  w \phi - K \phi_z = 0, \quad z = -H, z = 0 .
\end{equation}

Using a standard technique, separation of variables, let
$ \phi(z,t) = A(t) B(z) $. 
With $m = w/K$, this gives the following ordinary differential 
equation for $B$,
\begin{equation}\label{eq:B(z)}
  B'' - m B' + \lambda B = 0 ,
\end{equation}
or in Sturm-Liouville form, 
\begin{equation}
  (\e^{-mz} B')' + \lambda \e^{-mz} B = 0 .
\end{equation}
The boundary conditions are separated,
\begin{equation}
 B' = m B, \quad z = -H, z = 0.
\end{equation}
In this form we have a self-adjoint, regular Sturm-Liouville system.

To such a system, there is an increasing sequence of real eigenvalues,
$\lambda_0 < \lambda_1 < \dots$ with associated eigenfunctions
$B_n(z)$. These functions forms an orthogonal system with respect
to the inner-product 
\begin{equation}
  \langle f,g \rangle = \int_{-H}^0 f(z) g(z) \e^{-mz} dz .
\end{equation}



In our case the only eigenvalue $\lambda \le m^2/4$ is
$\lambda_0 = 0$ corresponding to the stationary
solution $\phi(z,t) = \e^{mz}$. For $\lambda > m^2/4$ 
let $\alpha^2 = \lambda - m^2/4$. 
The general solution to equation~(\ref{eq:B(z)}) is then
\begin{equation}
  B(z) = \e^{\half m z} (b_1 \cos(\alpha z) + b_2 \sin(\alpha z)),
\end{equation}
Imposing the boundary conditions gives
\begin{equation}
  2 \alpha b_2 = m b_1, \quad \mbox{and} \quad
   \alpha_n = \frac{n \pi}{H}, \quad n = 1, 2, \dots .
\end{equation}

The normalised eigenfunctions with respect to the inner product are
\begin{equation}
   \lambda_0 = 0, \quad B_0 = \sqrt{\frac{m}{1-\e^{-mH}}} \e^{mz}
\end{equation}
and for $n \ge 1$
\begin{equation}
   \lambda_n = \frac{m^2}{4} + \left(\frac{n \pi}{H}\right)^2, \quad
      B_n = (2 H \lambda_n)^{-\half} \e^{\half m z} 
            (2 \alpha_n \cos(\alpha_n z) + m \sin(\alpha_n z))
\end{equation}

For $\lambda_0 = 0$ the time dependent part $A_0$ is constant.
The time dependent part corresponding to $\lambda = \lambda_n$ with $n \ge
1$ is
\begin{equation}
   A_n(t) = \e^{-\lambda_n K t}
\end{equation}

This gives the general format for the solution of
equation~(\ref{eq:constpde}),
\begin{equation}
  \phi(z,t) = \sum_{n=0}^\infty a_n \e^{-\lambda_n K t} B_n(z)
\end{equation}
The coefficients $a_n$ are determined from the initial values
$\phi(z,0) = f(z)$ by
\begin{equation}
   a_n = \langle f, B_n \rangle = \int_{-H}^0 f(z) B_n(z) \e^{-mz} dz .
\end{equation}


\section{Linear conservative schemes}\label{app:lincons}

The linear finite difference schemes considered in
section~\ref{seq:linsch} are all in conservation form (\ref{eq:numlaw})
\begin{equation}\label{eq:appflux}
    (\phi_i^{n+1} - \phi_i^n) \Dz = (F_{i+1}^n - F_i^n) \Dt 
\end{equation}
where the flux function have the forms (\ref{eq:alphabeta})--(\ref{eq:ab})
\begin{equation}\label{eq:applin}
   F_i = \alpha_i \phi_i + b_i \phi_{i-1}
       = \frac{\Dz}{\Dt} ( a_i \phi_i + b_i \phi_{i-1} )
\end{equation}
where the coefficients are independent of the $\phi_i$-s.

\subsection{Conditions for consistence}\index{concistence}

A numerical scheme for a partial differential equation must be 
\emph{consistent}, this means that the scheme converges to the equation
as the time and space steps approach zero.

Here the first expression $F_i = \alpha_i \phi_i + \beta_i \phi_{i-1}$
is used. The $\phi_i$-s are defined as cell averages in 
equation \ref{eq:cellav}
For continuous functions  $\alpha(z)$, $\beta(z)$ and
$\phi(z)$, $F_i = \tilde{F}(z_i)$ where
\begin{equation}
  \tilde{F}(z) = \frac{1}{\Dz}
      \left( \alpha(z) \int_{z-\Dz}^z\!\phi(\zeta)\,d\zeta 
        +    \beta(z)  \int_z^{z+\Dz}\!\phi(\zeta)\,d\zeta \right) .
\end{equation}
The Taylor expansion of $\phi$ around z is
\begin{equation}
   \phi(\zeta) = \phi(z) + \phi_z(z)(\zeta-z) 
        + \frac{1}{2}\phi_{zz}(z)(\zeta-z)^2 + \cdots .
\end{equation}
Using this $\tilde{F}$ can be expanded as the series
\begin{equation}
  \tilde{F} = (\alpha + \beta) \phi 
       - \frac{1}{2} (\alpha - \beta) \phi_z \Dz
       + \frac{1}{6} (\alpha + \beta) \phi_{zz} \Dz^2 - +  \cdots .
\end{equation}
For consistence, $\tilde{F}$ must converge
to the flux function $F$,
\begin{equation} 
  \tilde{F} \rightarrow 
  F = w \phi - K \phi_z \quad \text{as $\Dz, \Dt \rightarrow 0$}.
\end{equation}
This gives the following consistence criterion 
\begin{alignat}{2}
     \alpha + \beta &\rightarrow w \quad 
                &\text{as $\Dt, \Dz \rightarrow 0$} \\
     (\alpha - \beta) \Dz &\rightarrow 2 K \quad 
                &\text{as $\Dt, \Dz \rightarrow  0$} 
\end{alignat}

The schemes considered in section~\ref{seq:linsch} should be consistent
by construction.  This is easily verified as $w = \alpha + \beta$ in all
three cases and \begin{equation} (\alpha - \beta)\Dz =
  \begin{cases}
    2 K  &  \text{FTCS}, \\
    2 K + w^2 \Dt & \text{Lax-Wendroff},  \\
    2 K + \abs{w} \Dz & \text{Upstream}.
  \end{cases}
\end{equation}



\subsection{Conditions for positivity and stability}\label{app:pos}
\index{positivity}\index{stability}

For this analysis the nondimensional form
\begin{equation}\label{eq:nondimflux}
   F_i = \frac{\Dz}{\Dt} ( a_i \phi_i + b_i \phi_{i-1} )
\end{equation}
of the numeric flux function is used.

Substituting equation \eqref{eq:nondimflux} and the boundary conditions
$F_i = F_{N+1} = 0$ in \eqref{eq:appflux} gives the following expession
for the solution $\phi^+$ at the next time step 
\begin{align}
  \phi_1^+ & = (1 + b_2) \phi_1 + a_2 \phi_2, \\
  \phi_i^+ & = - b_i \phi_{i-1} + (1 - a_i + b_{i+1}) \phi_i 
             + a_{i+1} \phi_{i+1} , \quad i = 2, \dots, N-1, 
              \label{eq:aftersubst2} \\
  \phi_N^+ & = -b_N \phi_{N-1} + (1 - a_N) \phi_N .             
\end{align}


For the scheme to be positive we must have $\phi_i^+ \ge 0$ for all
nonnegative values of $\phi_{i-1}, \phi_i, \phi_{i+1}$.
In other words, the coefficients above must all be nonnegative
\begin{align}
   a_{i} & \ge 0, \quad i = 2, \dots, N, \label{eq:poscond1}\\
   b_i   & \le 0, \quad i = 2, \dots, N, \\
   - b_2 & \le 1, \label{eq:poscond3} ,\\ 
   a_i - b_{i+1} & \le 1, \quad i = 2, \dots, N-1, \\
   a_N           & \le 1 . \label{eq:poscond5} 
\end{align}

If the coefficients $a$ and $b$ are constant, a traditional Von Neumann
stability analysis can be carried out. The amplification factor is
\begin{equation}
  \lambda = 1-(a-b)\cos^2\Theta  + \i (a+b)\sin \Theta
\end{equation}
for $\Theta = m \pi \Dz$. The stability condition $\abs{\lambda} \le 1$
for all values of $\Theta$ then becomes
\begin{equation}
  (a+b)^2 \le (a-b) \le 1 .
\end{equation}
In this case, the positivity conditions,
eqs.~(\ref{eq:poscond1}--\ref{eq:poscond5}) are reduced to
\begin{equation}
  a \ge 0, \quad b \le 0, \quad a - b \le 1 ,
\end{equation}
which is is stronger than the stability condition.

\subsection{Positivity with loss term}\label{app:posloss}

Egg production works to enhance the positivity of the solution
while loss of eggs may ruin an otherwise positive solution.
For the analysis of this situation we use equation~(\ref{eq:numlaw})
with $F_i$ given by the nondimensional form (\ref{eq:ab}) and the
source term $Q_i$ given by eq.~(\ref{eq:kildediskret}).
Substitutitint the $F$ and $Q$ terms in (\ref{eq:numlaw}) gives the
following genrealisation of eq.~(\ref{eq:aftersubst2})
\begin{equation}
  \phi_i^+  = P_i \frac{1-\e^{-\alpha_i \Dt}}{\alpha_i}
          - b_i \phi_{i-1} + (\e^{-\alpha_i \Dt} - a_i + b_{i+1}) \phi_i 
             + a_{i+1} \phi_{i+1} , \quad i = 2, \dots, N-1 .
\end{equation}
If this should be positive, regardless of $P_i$ the  conditions
(\ref{eq:poscond3}--\ref{eq:poscond5}) must be strengthened
\begin{align}
   - b_2 & \le \e^{-\alpha_i \Dt}, \\ 
   a_i - b_{i+1} & \le \e^{-\alpha_i \Dt}, \quad i = 2, \dots, N-1, \\
   a_N           & \le \e^{-\alpha_i \Dt} . 
\end{align}






\chapter{Future work}

Hopefully, as the toolbox is put to use, it will develop to better
suit the need of scientists in the field. Below is an unsorted list
of things that may be included.

\begin{itemize}
  \item Better numerical schemes for the convection-diffusion equation.
  \item Implementation of Lagrangian (particle tracking) approach.
  \item Implement analytical transient solution of
        convection-diffusion equation with constant coefficients.
  \item More functions for analysis of observed egg distributions.
  \item Non-uniform vertical grid.
  \item Ekman solver.
  \item Turbulence model(s).
  \item Homepage for VertEgg on World Wide Web.
\end{itemize}

